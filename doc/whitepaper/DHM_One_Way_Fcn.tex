\documentclass[11pt]{article}
%Gummi|065|=)
\title{\textbf{Diffie Hellman Merkle One-Way functiion}}
\author{Pratyush Bhatt}
\date{}
\begin{document}

\begin{equation}
f(g, s, p) = g ^ s \bmod p
\end{equation}
Where:\\
mod is the modulus (remainder) operator\\
 p is a prime number, and\\
 g is a number chosen from a special group of numbers, related to
   the prime number p, for which a particular relation holds true
   (more on this later).
\end{document}
