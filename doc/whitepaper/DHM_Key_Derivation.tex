\documentclass[11pt]{article}
\usepackage{amsmath}
%Gummi|065|=)
\title{\textbf{DHM Key Derivation}}
\author{Pratyush Bhatt}
\date{}
\begin{document}
For two endpoints, Max and Ned,\\

let:\\
\(p\) be a prime number\\
\(g\) be a primitve root, modulo \(p\)\\
\(s_{m}\) be Max's secret\\
\(s_{n}\) be Ned's secret\\
\(f_{m}\) be Max's secret, obfuscated by the one-way function.\\
\(f_{n}\) be Ned's secret, obfuscated by the one-way function.\\
\(k_{m}\) be the shared key computed by the Max\\
\(k_{n}\) be the shared key computed by the Ned\\

then:\\
\begin{equation}
\begin{aligned}
k_{m} &= f( f_{n}, s_{m}, p)\\
 & = f( f(g, s_{n}, p), s_{m}, p)\\
 & = (g^{s_{n}} \bmod p) ^ {s_{m}} \bmod p 
\end{aligned}
\end{equation}
\begin{equation}
\begin{aligned}
k_{n} &= f( f_{m}, s_{n}, p)\\
 & = f( f(g, s_{m}, p), s_{n}, p)\\
 & = (g^{s_{m}} \bmod p) ^ {s_{n}} \bmod p 
\end{aligned}
\end{equation}
Recall that the number \(g\) was specially chosen for the number \(p\). It was chosen as a "primitive root, modulo \(p\)" which really just means that it was chosen such that:
\begin{equation}
 (g ^ {s_{m}} \bmod p) ^ {s_{n}} \bmod p = (g ^ {s_{n}} \bmod p) ^ {s_{m}} \bmod p
\end{equation}
holds true for all values of \(s_{m}\) and \(s_{n}\)\\
so \(k_{m} = k_{n}\)
\end{document}
